\documentclass[12pt, letterpaper]{article}
\usepackage{SyllabusStyle}

\begin{document}
\begin{center}
	{\Large \textsc{Physical Chemistry Lab I}}

	CHEM 3615
\end{center}

\begin{center}
	{\large Fall 2024}
\end{center}
\begin{center}
	\rule{0.99\textwidth}{0.4pt}
	\begin{tabular}{llcll}
		\textbf{Instructor:} & Matthew Rowley           &  & \textbf{Office Hours:} & Daily 10:00 am -- 11:00 am \\
		\textbf{Telephone:}  & (435) 586-7875           &  &                        &  \\
		\textbf{Email:}      & matthewrowley$1$@suu.edu &  & \textbf{Office:}       & SC-220                   \\
		\multicolumn{5}{c}{Please include the course number in the subject line of all correspondence.}
	\end{tabular}
	\rule{0.99\textwidth}{0.4pt}
\end{center}


\section*{Course Description}
This course is the laboratory to accompany CHEM 3610 -- Physical Chemistry I. We will observe and explore chemical systems which clearly demonstrate principles of thermodynamics, kinetics, and statistical mechanics

\paragraph{Prerequisites:}
None

\paragraph{Concurrent requisite:}
CHEM 3610 -- Physical Chemistry I

\paragraph{Course Materials:} ~

\noindent No lab manual will be required. The instructions for each experiment will be posted on Canvas.

\noindent You will be required to have and wear your own pair of OSHA-approved safety goggles whenever you are in the lab. Students without eye protection will be required to leave the lab and will receive a zero for the labwork that day.

\paragraph{Student Learning Outcomes:}
\begin{description}
	\item[Knowledge of the physical and natural world] -- Students will recall, interpret, compare, explain, and apply chemistry terminology and theory.
	\item[Quantitative Literacy] -- Students will use chemical equations, graphs and tables to interpret and communicate chemical information.
	\item[Inquiry and Analysis] -- Students will solve complex chemical problems.
	\item[Communication] -- Students will report laboratory results clearly and concisely.
	\item[Problem Solving] -- Students will design and implement experimental procedures.
	\item[Teamwork] -- Students will productively interact with each other to successfully conduct chemistry experiments.
\end{description}

\section*{Laboratory Work}
Before lab, you are expected to have read the handout of your experiment as well as review your lecture notes from class. Come prepared to enter your data into the lab computers and have a USB drive with you. You may perform each laboratory with a lab partner and you may acquire your data together during your scheduled lab time. However, you must NOT work with your lab partner beyond this. All analysis of data and calculations as well as all laboratory reports must be done on an individual basis. Failure to do so will result in a zero for the lab in question.

\noindent Please follow all safety procedures, especially by wearing safety glasses or goggles. When leaving the lab, please make sure it is in the same condition as it was when you arrived. Be respectful of others.

\paragraph{Laboratory Risk:}
Chemical exposure is a constant risk in a chemistry lab. To minimize the risk to yourself and those around you, the following rules must be followed:
\begin{itemize}
	\item Never taste or smell a chemical or pipette by mouth.
	\item Wash your hands before leaving the lab and frequently during the lab to avoid accidental contamination of yourself and others.
	\item Dispose of chemicals only as directed. Nothing goes down the sink unless expressly directed.
	\item Keep your work area clean; wipe up any spills (liquid or solid) immediately.
	\item Replace caps on reagent bottles, and never return chemicals to the original container.
	\item No shorts, tank tops, or sandals allowed in lab, and long hair should be restrained.
	\item Wear safety glasses at all times when in the lab.
\end{itemize}
Students enrolling in this course should realize that they are voluntarily exposing themselves to a variety of chemicals, some of which could be irritating or hazardous with excessive exposure.  For those persons with a sensitive medical condition such as allergies, precautions such as wearing additional protective garments, delaying enrolling, or even not enrolling in a class may be necessary.  In particular, women who are their first trimester of pregnancy should avoid exposure to many chemicals unless approved by their physician.

\section*{Tentative Schedule}
This class will meet on Tuesdays from 4:00pm -- 6:50pm in room 224 of the Science Center (SC)

\noindent
$\star$ Remember the literature report assignment which you should work on \emph{throughout} the semester!

\paragraph{Week 1: Aug. 28 -- Aug. 30}~

No Lab

\paragraph{Week 2: Sep. 2 -- Sep. 6}~

\begin{itemize}
  \item Group 1: Adiabatic Expansion of Gases
  \item Group 2: Measurement of the 2nd Virial Coefficient
  \item Group 3: Determination of the Caloric Content of Food
\end{itemize}

\paragraph{Week 3: Sep. 9 -- Sep. 13}~

Continue labs and work on lab reports

\paragraph{Week 4: Sep. 16 -- Sep. 20}~

\begin{itemize}
  \item Group 1: Determination of the Caloric Content of Food
  \item Group 2: Adiabatic Expansion of Gases
  \item Group 3: Measurement of the 2nd Virial Coefficient
\end{itemize}

\paragraph{Week 5: Sep. 23 -- Sep. 27}~

Continue labs and work on lab reports

\paragraph{Week 6: Sep. 30 -- Oct. 4}~

\begin{itemize}
  \item Group 1: Measurement of the 2nd Virial Coefficient
  \item Group 2: Determination of the Caloric Content of Food
  \item Group 3: Adiabatic Expansion of Gases
\end{itemize}

\paragraph{Week 7: Oct. 7 -- Oct. 11}~

Continue labs and work on lab reports

\paragraph{Week 8: Oct. 14 -- Oct. 18}~

\textbf{No Classes this week (Fall Break!)}

\paragraph{Week 9: Oct. 21 -- Oct. 25}~

\begin{itemize}
  \item Group 1: Measurement of Joule-Thompson Coefficients
  \item Group 2: Computational Study on the Enthalpy of Formation
  \item Group 3: Kinetics of the Iodination of Acetone
\end{itemize}

\paragraph{Week 10: Oct. 28 -- Nov. 1}~

Continue labs and work on lab reports

\paragraph{Week 11: Nov. 4 -- Nov. 8}~

\begin{itemize}
  \item Group 1: Kinetics of the Iodination of Acetone
  \item Group 2: Measurement of Joule-Thompson Coefficients
  \item Group 3: Computational Study on the Enthalpy of Formation
\end{itemize}

\paragraph{Week 12: Nov. 11 -- Nov. 15}~

Continue labs and work on lab reports

\paragraph{Week 13: Nov. 18 -- Nov. 22}~

\begin{itemize}
  \item Group 1: Computational Study on the Enthalpy of Formation
  \item Group 2: Kinetics of the Iodination of Acetone
  \item Group 3: Measurement of Joule-Thompson Coefficients
\end{itemize}

\paragraph{Week 14: Nov. 25 -- Nov. 29}~

\textbf{No Classes this week (Thanksgiving Break!)}


\paragraph{Week 15: Dec. 2 -- Dec. 6}~

\textbf{Final Exam}

\paragraph{Finals Week}~

No Final -- You took it last week!

\section*{Course Requirements}
Grades for this class will be determined based on the following items:

\begin{description}
	\item[Pre-Lab Quizzes (10 points each)] -- Quizzes must be completed at the beginning of each lab. You may take the better score out of two attempts at these quizzes.
	\item[Lab Reports (40 points each)] -- Reports must be turned in at the beginning of the following lab.
	\item[Lab Final (100 points)] -- The final will be given on the last scheduled day of class (Dec. 1).
\end{description}

\noindent Final Grades will be assigned according to the following scale:

\begin{tabular}{cl|c|cl}
	Percentage & Grade &  & Percentage & Grade \\ \midrule
	100--93.0  & A     &  & 77.0--73.0 & C     \\
	93.0--90.0 & A-    &  & 73.0--70.0 & C-    \\
	90.0--87.0 & B+    &  & 70.0--67.0 & D+    \\
	87.0--83.0 & B     &  & 67.0--63.0 & D     \\
	83.0--80.0 & B-    &  & 63.0--60.0 & D-    \\
	80.0--77.0 & C+    &  & < 60.0     & F
\end{tabular}

\paragraph{Note that you must complete \emph{all} of the labs to pass this course!} Regardless of your other scores, an incomplete lab will result in an incomplete grade for entire course.

\paragraph{Report Grading:}
Lab reports will be graded on the quality of both their \emph{scientific content} and their \emph{presentation} in the following way:
\begin{description}
	\item[Presentation] -- Presentation includes writing quality, writing style, clarity, organization, formatting, grammar, etc.

	\item[Scientific Content] -- Your report should demonstrate a clear understanding of the basic principles at play in the lab. Data presentation should show an understanding of what the data mean and why they  are important (e.g. mislabeled axes show a lack of understanding). Analysis of your data and any conclusions drawn should have a sound basis in the scientific theories you have been taught.
\end{description}

Different chemistry journals have different formats and requirements. To keep things simple, your reports should have four sections:
\begin{description}
	\item[Introduction] -- Show background knowledge and general understanding of the experiment.
	\item[Method] -- Outline your specific procedure with enough detail that a competent chemist could reproduce your work.
	\item[Results] -- Results should include charts of any relevant data, as well as a description of any qualitative observations made in the course of the experiment.
	\item[Discussion] -- Interpret your results and draw conclusions. The lab manual will often prompt you with questions which must be answered in this section.
\end{description}

\paragraph{Late Work Policy:}
Laboratory reports must be turned in at or before the beginning of the following lab. There is a large window of time in which to analyze and write-up your results, so please plan to do the work early if you have any scheduling conflicts. Late work will not be accepted.

\paragraph{Make-up Work Policy:}
In general, there will be no opportunity to make up missed labs. This is particularly important because you must complete all of the labs to pass this course. If you must miss your assigned lab time, please arrange with your partner and me to do the lab some other time within the same week.


\section*{Miscellany}

\paragraph{Scientific Calculator:}
There are many different ways to calculate figures during homework. It is tempting to rely on Online resources such as \href{http://www.wolframalpha.com}{http://www.wolframalpha.com}, or to simply use a calculator application on a smartphone. During exams, however, any devices capable of connecting to the Internet will \emph{not} be allowed. You will instead need a scientific calculator capable of performing exponentiation and logarithms for the exams. Using this calculator exclusively while doing homework will ensure that you are proficient with it for use during exams.

\paragraph{Academic Credit:}
According to the federal definition of a Carnegie credit hour: A credit hour of work is the equivalent of approximately 60 minutes of class time or independent study work. A minimum of 45 hours of work by each student is required for each unit of credit. Credit is earned only when course requirements are met. One (1) credit hour is equivalent to 15 contact hours of lecture, discussion, testing, evaluation, or seminar, as well as 30 hours of student homework. An equivalent amount of work is expected for laboratory work, internships, practica, studio, and other academic work leading to the awarding of credit hours. Credit granted for individual courses, labs, or studio classes range from 0.5 to 15 credit hours per semester.

\paragraph{Academic Freedom:}
SUU is operated for the common good of the greater community it serves. The common good depends upon the free search for truth and its free exposition. Academic Freedom is the right of faculty to study, discuss, investigate, teach, and publish. Academic Freedom is essential to these purposes and applies to both teaching and research. 

\noindent
Academic Freedom in the realm of teaching is fundamental for the protection of the rights of the faculty member and of you, the student, with respect to the free pursuit of learning and discovery. Faculty members possess the right to full freedom in the classroom in discussing their subjects. They may present any controversial material relevant to their courses and their intended learning outcomes, but they shall take care not to introduce into their teaching controversial materials which have no relation to the subject being taught or the intended learning outcomes for the course.

\noindent
As such, students enrolled in any course at SUU may encounter topics, perspectives, and ideas that are unfamiliar or controversial, with the educational intent of providing a meaningful learning environment that fosters your growth and development. These parameters related to Academic Freedom are included in SUU \href{https://www.suu.edu/policies/06/06.html}{Policy 6.6}.

\paragraph{Academic Integrity:}
Scholastic honesty is expected of all students. Dishonesty will not be tolerated and will be prosecuted to the fullest extent (see \href{https://www.suu.edu/policies/06/33.html}{SUU Policy 6.33}). You are expected to have read and understood the current SUU student conduct code (\href{https://www.suu.edu/policies/11/02.html}{SUU Policy 11.2}) regarding student responsibilities and rights, the intellectual property policy (\href{https://www.suu.edu/policies/05/52.html}{SUU Policy 5.52}), information about procedures, and what constitutes acceptable behavior. 

\noindent
\underline{Please Note}: The use of websites or services that sell essays is a violation of these policies; likewise, the use of websites or services that provide answers to assignments, quizzes, or tests is also a violation of these policies. Regarding the use of Generative Artificial Intelligence (AI), you should check with your individual course instructor.

\paragraph{ADA Statement:}
Students with medical, psychological, learning, or other disabilities desiring academic adjustments, accommodations, or auxiliary aids will need to contact the \href{https://www.suu.edu/disabilityservices/}{Disability Resource Center}, located in Room 206F of the Sharwan Smith Center or by phone at (435) 865-8042. The Disability Resource Center determines eligibility for and authorizes the provision of services.

\noindent
If your instructor requires attendance, you may need to seek an ADA accommodation to request an exception to this attendance policy. Please contact the Disability Resource Center to determine what, if any, ADA accommodations are reasonable and appropriate.

\paragraph{Non-Discrimination Statement:}
SUU is committed to fostering an inclusive community of lifelong learners and believes our university's encompassing of different views, beliefs, and identities makes us stronger, more innovative, and better prepared for the global society. 

\noindent
SUU does not discriminate on the basis of race, religion, color, national origin, citizenship, sex (including sex discrimination and sexual harassment), sexual orientation, gender identity, age, ancestry, disability status, pregnancy, pregnancy-related conditions, genetic information, military status, veteran status, or other bases protected by applicable law in employment, treatment, admission, access to educational programs and activities, or other University benefits or services.

\noindent
SUU strives to cultivate a campus environment that encourages freedom of expression from diverse viewpoints. We encourage all to dialogue within a spirit of respect, civility, and decency. 

\noindent
For additional information on non-discrimination, please see \href{https://www.suu.edu/policies/05/27.html}{Policy 5.27} and/or visit:\newline \href{https://www.suu.edu/nondiscrimination.}{https://www.suu.edu/nondiscrimination.}

\paragraph{Pregnancy:}
Students who are or become pregnant during this course may receive reasonable modifications to facilitate continued access and participation in the course. Pregnancy and related conditions are broadly defined to include pregnancy, childbirth, termination of pregnancy, lactation, related medical conditions, and recovery. To obtain reasonable modifications, please make a request to: \href{mailto:title9@suu.edu}{title9@suu.edu}. To learn more visit: \href{https://www.suu.edu/titleix/pregnancy.html}{https://www.suu.edu/titleix/pregnancy.html}.

\paragraph{Mandatory Reporting:}
University policy (\href{https://www.suu.edu/policies/05/60.html}{SUU Policy 5.60}) requires instructors to report disclosures received from students that indicate they have been subjected to sexual misconduct/harassment. The University defines sexual harassment consistent with Federal Regulations (\href{https://www.ecfr.gov/current/title-34/subtitle-B/chapter-I/part-106/subpart-D}{34 C.F.R. Part 106, Subpart D}) to include quid pro quo, hostile environment harassment, sexual assault, dating violence, domestic violence, and stalking. When students communicate this information to an instructor in-person, by email, or within writing assignments, the instructor will report that to the Title IX Coordinator to ensure students receive support from the Title IX Office. A reporting form is available at \href{https://cm.maxient.com/reportingform.php?SouthernUtahUniv}{https://cm.maxient.com/reportingform.php?SouthernUtahUniv}

\paragraph{Emergency Management Statement:}
In case of emergency, the university's Emergency Notification System (ENS) will be activated. Students are encouraged to maintain updated contact information using the link on the homepage of the \emph{mySUU} portal. In addition, students are encouraged to familiarize themselves with the Emergency Response Protocols posted in each classroom. Detailed information about the university's emergency management plan can be found at: \href{http://www.suu.edu/emergency}{http://www.suu.edu/emergency}

\paragraph{HEOA Compliance Statement:}
For a full set of Higher Education Opportunity Act (HEOA) compliance statements, please visit \href{https://www.suu.edu/heoa}{https://www.suu.edu/heoa}. The sharing of copyrighted material through peer-to-peer (P2P) file sharing, except as provided under U.S. copyright law, is prohibited by law; additional information can be found at \newline\href{https://my.suu.edu/help/article/1096/heoa-compliance-plan}{https://my.suu.edu/help/article/1096/heoa-compliance-plan}.

\noindent
You are also expected to comply with policies regarding intellectual property (\href{https://www.suu.edu/policies/05/52.html}{SUU Policy 5.52}) and copyright (\href{https://www.suu.edu/policies/05/54.html}{SUU Policy 5.54}).

\paragraph{SUUSA Statement:}
As a student at SUU, you have representation from the SUU Student Association (SUUSA) which advocates for student interests and helps work as a liaison between the students and the university administration. You can submit MySUU Voice feedback by going to \href{https://www.suu.edu/suusa/voice}{https://www.suu.edu/suusa/voice}. Likewise, you can learn more about SUUSA’s Executive Council at \href{https://www.suu.edu/suusa/executive-council}{https://www.suu.edu/suusa/executive-council} and about all of SUUSA’s Student Senators at \href{https://www.suu.edu/suusa/senate}{https://www.suu.edu/suusa/senate}. If you have any specific concerns regarding any of your courses, please contact the SUUSA VP of Academics at: \href{suusa_academicsvp@suu.edu}{suusa\_\ignorespaces academicsvp@suu.edu}.

\paragraph{Thriving Thunderbirds:}
Mental health is essential for your academic success. If you are struggling with mental health issues, SUU provides resources, support, and services to help you. Please visit \href{https://www.suu.edu/mentalhealth}{https://www.suu.edu/mentalhealth} for access to these valuable resources.

\noindent
If you need assistance navigating any of the resources, please contact \href{https://www.suu.edu/caps/}{Counseling and Psychological Services}, the \href{https://www.suu.edu/deanofstudents/}{Dean of Students’ Office}, or the \href{https://www.suu.edu/health/}{Health and Wellness Center}.

\paragraph{Land Acknowledgement Statement:}
SUU wishes to acknowledge and honor the Indigenous communities of this region as original possessors, stewards, and inhabitants of this Too’veep (land), and recognize that the University is situated on the traditional homelands of the Nung’wu (Southern Paiute People). We recognize that these lands have deeply rooted spiritual, cultural, and historical significance to the Southern Paiutes. We offer gratitude for the land itself, for the collaborative and resilient nature of the Southern Paiute people, and for the continuous opportunity to study, learn, work, and build community on their homelands here today.

\paragraph{Disclaimer:}
Information contained in this syllabus, other than the grading, late assignments, make up work and attendance policies, may be subject to change with advance notice, as deemed appropriate by the instructor.

\end{document}
